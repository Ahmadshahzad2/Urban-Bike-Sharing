% Options for packages loaded elsewhere
\PassOptionsToPackage{unicode}{hyperref}
\PassOptionsToPackage{hyphens}{url}
%
\documentclass[
]{article}
\usepackage{amsmath,amssymb}
\usepackage{iftex}
\ifPDFTeX
  \usepackage[T1]{fontenc}
  \usepackage[utf8]{inputenc}
  \usepackage{textcomp} % provide euro and other symbols
\else % if luatex or xetex
  \usepackage{unicode-math} % this also loads fontspec
  \defaultfontfeatures{Scale=MatchLowercase}
  \defaultfontfeatures[\rmfamily]{Ligatures=TeX,Scale=1}
\fi
\usepackage{lmodern}
\ifPDFTeX\else
  % xetex/luatex font selection
\fi
% Use upquote if available, for straight quotes in verbatim environments
\IfFileExists{upquote.sty}{\usepackage{upquote}}{}
\IfFileExists{microtype.sty}{% use microtype if available
  \usepackage[]{microtype}
  \UseMicrotypeSet[protrusion]{basicmath} % disable protrusion for tt fonts
}{}
\makeatletter
\@ifundefined{KOMAClassName}{% if non-KOMA class
  \IfFileExists{parskip.sty}{%
    \usepackage{parskip}
  }{% else
    \setlength{\parindent}{0pt}
    \setlength{\parskip}{6pt plus 2pt minus 1pt}}
}{% if KOMA class
  \KOMAoptions{parskip=half}}
\makeatother
\usepackage{xcolor}
\usepackage[margin=1in]{geometry}
\usepackage{color}
\usepackage{fancyvrb}
\newcommand{\VerbBar}{|}
\newcommand{\VERB}{\Verb[commandchars=\\\{\}]}
\DefineVerbatimEnvironment{Highlighting}{Verbatim}{commandchars=\\\{\}}
% Add ',fontsize=\small' for more characters per line
\usepackage{framed}
\definecolor{shadecolor}{RGB}{248,248,248}
\newenvironment{Shaded}{\begin{snugshade}}{\end{snugshade}}
\newcommand{\AlertTok}[1]{\textcolor[rgb]{0.94,0.16,0.16}{#1}}
\newcommand{\AnnotationTok}[1]{\textcolor[rgb]{0.56,0.35,0.01}{\textbf{\textit{#1}}}}
\newcommand{\AttributeTok}[1]{\textcolor[rgb]{0.13,0.29,0.53}{#1}}
\newcommand{\BaseNTok}[1]{\textcolor[rgb]{0.00,0.00,0.81}{#1}}
\newcommand{\BuiltInTok}[1]{#1}
\newcommand{\CharTok}[1]{\textcolor[rgb]{0.31,0.60,0.02}{#1}}
\newcommand{\CommentTok}[1]{\textcolor[rgb]{0.56,0.35,0.01}{\textit{#1}}}
\newcommand{\CommentVarTok}[1]{\textcolor[rgb]{0.56,0.35,0.01}{\textbf{\textit{#1}}}}
\newcommand{\ConstantTok}[1]{\textcolor[rgb]{0.56,0.35,0.01}{#1}}
\newcommand{\ControlFlowTok}[1]{\textcolor[rgb]{0.13,0.29,0.53}{\textbf{#1}}}
\newcommand{\DataTypeTok}[1]{\textcolor[rgb]{0.13,0.29,0.53}{#1}}
\newcommand{\DecValTok}[1]{\textcolor[rgb]{0.00,0.00,0.81}{#1}}
\newcommand{\DocumentationTok}[1]{\textcolor[rgb]{0.56,0.35,0.01}{\textbf{\textit{#1}}}}
\newcommand{\ErrorTok}[1]{\textcolor[rgb]{0.64,0.00,0.00}{\textbf{#1}}}
\newcommand{\ExtensionTok}[1]{#1}
\newcommand{\FloatTok}[1]{\textcolor[rgb]{0.00,0.00,0.81}{#1}}
\newcommand{\FunctionTok}[1]{\textcolor[rgb]{0.13,0.29,0.53}{\textbf{#1}}}
\newcommand{\ImportTok}[1]{#1}
\newcommand{\InformationTok}[1]{\textcolor[rgb]{0.56,0.35,0.01}{\textbf{\textit{#1}}}}
\newcommand{\KeywordTok}[1]{\textcolor[rgb]{0.13,0.29,0.53}{\textbf{#1}}}
\newcommand{\NormalTok}[1]{#1}
\newcommand{\OperatorTok}[1]{\textcolor[rgb]{0.81,0.36,0.00}{\textbf{#1}}}
\newcommand{\OtherTok}[1]{\textcolor[rgb]{0.56,0.35,0.01}{#1}}
\newcommand{\PreprocessorTok}[1]{\textcolor[rgb]{0.56,0.35,0.01}{\textit{#1}}}
\newcommand{\RegionMarkerTok}[1]{#1}
\newcommand{\SpecialCharTok}[1]{\textcolor[rgb]{0.81,0.36,0.00}{\textbf{#1}}}
\newcommand{\SpecialStringTok}[1]{\textcolor[rgb]{0.31,0.60,0.02}{#1}}
\newcommand{\StringTok}[1]{\textcolor[rgb]{0.31,0.60,0.02}{#1}}
\newcommand{\VariableTok}[1]{\textcolor[rgb]{0.00,0.00,0.00}{#1}}
\newcommand{\VerbatimStringTok}[1]{\textcolor[rgb]{0.31,0.60,0.02}{#1}}
\newcommand{\WarningTok}[1]{\textcolor[rgb]{0.56,0.35,0.01}{\textbf{\textit{#1}}}}
\usepackage{longtable,booktabs,array}
\usepackage{calc} % for calculating minipage widths
% Correct order of tables after \paragraph or \subparagraph
\usepackage{etoolbox}
\makeatletter
\patchcmd\longtable{\par}{\if@noskipsec\mbox{}\fi\par}{}{}
\makeatother
% Allow footnotes in longtable head/foot
\IfFileExists{footnotehyper.sty}{\usepackage{footnotehyper}}{\usepackage{footnote}}
\makesavenoteenv{longtable}
\usepackage{graphicx}
\makeatletter
\def\maxwidth{\ifdim\Gin@nat@width>\linewidth\linewidth\else\Gin@nat@width\fi}
\def\maxheight{\ifdim\Gin@nat@height>\textheight\textheight\else\Gin@nat@height\fi}
\makeatother
% Scale images if necessary, so that they will not overflow the page
% margins by default, and it is still possible to overwrite the defaults
% using explicit options in \includegraphics[width, height, ...]{}
\setkeys{Gin}{width=\maxwidth,height=\maxheight,keepaspectratio}
% Set default figure placement to htbp
\makeatletter
\def\fps@figure{htbp}
\makeatother
\setlength{\emergencystretch}{3em} % prevent overfull lines
\providecommand{\tightlist}{%
  \setlength{\itemsep}{0pt}\setlength{\parskip}{0pt}}
\setcounter{secnumdepth}{-\maxdimen} % remove section numbering
\ifLuaTeX
  \usepackage{selnolig}  % disable illegal ligatures
\fi
\IfFileExists{bookmark.sty}{\usepackage{bookmark}}{\usepackage{hyperref}}
\IfFileExists{xurl.sty}{\usepackage{xurl}}{} % add URL line breaks if available
\urlstyle{same}
\hypersetup{
  pdftitle={Project},
  pdfauthor={Ahmad Shahzad},
  hidelinks,
  pdfcreator={LaTeX via pandoc}}

\title{Project}
\author{Ahmad Shahzad}
\date{2023-11-21}

\begin{document}
\maketitle

\begin{Shaded}
\begin{Highlighting}[]
\FunctionTok{library}\NormalTok{(readr)   }\CommentTok{\# For reading the data}
\FunctionTok{library}\NormalTok{(dplyr)   }\CommentTok{\# For data manipulation}
\end{Highlighting}
\end{Shaded}

\begin{verbatim}
## 
## Attaching package: 'dplyr'
\end{verbatim}

\begin{verbatim}
## The following objects are masked from 'package:stats':
## 
##     filter, lag
\end{verbatim}

\begin{verbatim}
## The following objects are masked from 'package:base':
## 
##     intersect, setdiff, setequal, union
\end{verbatim}

\begin{Shaded}
\begin{Highlighting}[]
\FunctionTok{library}\NormalTok{(knitr)}
\NormalTok{dataset }\OtherTok{\textless{}{-}} \FunctionTok{read\_csv}\NormalTok{(}\StringTok{"day.csv"}\NormalTok{)}
\end{Highlighting}
\end{Shaded}

\begin{verbatim}
## Rows: 731 Columns: 16
## -- Column specification --------------------------------------------------------
## Delimiter: ","
## dbl  (15): instant, season, yr, mnth, holiday, weekday, workingday, weathers...
## date  (1): dteday
## 
## i Use `spec()` to retrieve the full column specification for this data.
## i Specify the column types or set `show_col_types = FALSE` to quiet this message.
\end{verbatim}

\begin{Shaded}
\begin{Highlighting}[]
\FunctionTok{kable}\NormalTok{(}\FunctionTok{head}\NormalTok{(dataset))}
\end{Highlighting}
\end{Shaded}

\begin{longtable}[]{@{}
  >{\raggedleft\arraybackslash}p{(\columnwidth - 30\tabcolsep) * \real{0.0606}}
  >{\raggedright\arraybackslash}p{(\columnwidth - 30\tabcolsep) * \real{0.0833}}
  >{\raggedleft\arraybackslash}p{(\columnwidth - 30\tabcolsep) * \real{0.0530}}
  >{\raggedleft\arraybackslash}p{(\columnwidth - 30\tabcolsep) * \real{0.0227}}
  >{\raggedleft\arraybackslash}p{(\columnwidth - 30\tabcolsep) * \real{0.0379}}
  >{\raggedleft\arraybackslash}p{(\columnwidth - 30\tabcolsep) * \real{0.0606}}
  >{\raggedleft\arraybackslash}p{(\columnwidth - 30\tabcolsep) * \real{0.0606}}
  >{\raggedleft\arraybackslash}p{(\columnwidth - 30\tabcolsep) * \real{0.0833}}
  >{\raggedleft\arraybackslash}p{(\columnwidth - 30\tabcolsep) * \real{0.0833}}
  >{\raggedleft\arraybackslash}p{(\columnwidth - 30\tabcolsep) * \real{0.0682}}
  >{\raggedleft\arraybackslash}p{(\columnwidth - 30\tabcolsep) * \real{0.0682}}
  >{\raggedleft\arraybackslash}p{(\columnwidth - 30\tabcolsep) * \real{0.0682}}
  >{\raggedleft\arraybackslash}p{(\columnwidth - 30\tabcolsep) * \real{0.0758}}
  >{\raggedleft\arraybackslash}p{(\columnwidth - 30\tabcolsep) * \real{0.0530}}
  >{\raggedleft\arraybackslash}p{(\columnwidth - 30\tabcolsep) * \real{0.0833}}
  >{\raggedleft\arraybackslash}p{(\columnwidth - 30\tabcolsep) * \real{0.0379}}@{}}
\toprule\noalign{}
\begin{minipage}[b]{\linewidth}\raggedleft
instant
\end{minipage} & \begin{minipage}[b]{\linewidth}\raggedright
dteday
\end{minipage} & \begin{minipage}[b]{\linewidth}\raggedleft
season
\end{minipage} & \begin{minipage}[b]{\linewidth}\raggedleft
yr
\end{minipage} & \begin{minipage}[b]{\linewidth}\raggedleft
mnth
\end{minipage} & \begin{minipage}[b]{\linewidth}\raggedleft
holiday
\end{minipage} & \begin{minipage}[b]{\linewidth}\raggedleft
weekday
\end{minipage} & \begin{minipage}[b]{\linewidth}\raggedleft
workingday
\end{minipage} & \begin{minipage}[b]{\linewidth}\raggedleft
weathersit
\end{minipage} & \begin{minipage}[b]{\linewidth}\raggedleft
temp
\end{minipage} & \begin{minipage}[b]{\linewidth}\raggedleft
atemp
\end{minipage} & \begin{minipage}[b]{\linewidth}\raggedleft
hum
\end{minipage} & \begin{minipage}[b]{\linewidth}\raggedleft
windspeed
\end{minipage} & \begin{minipage}[b]{\linewidth}\raggedleft
casual
\end{minipage} & \begin{minipage}[b]{\linewidth}\raggedleft
registered
\end{minipage} & \begin{minipage}[b]{\linewidth}\raggedleft
cnt
\end{minipage} \\
\midrule\noalign{}
\endhead
\bottomrule\noalign{}
\endlastfoot
1 & 2011-01-01 & 1 & 0 & 1 & 0 & 6 & 0 & 2 & 0.344167 & 0.363625 &
0.805833 & 0.1604460 & 331 & 654 & 985 \\
2 & 2011-01-02 & 1 & 0 & 1 & 0 & 0 & 0 & 2 & 0.363478 & 0.353739 &
0.696087 & 0.2485390 & 131 & 670 & 801 \\
3 & 2011-01-03 & 1 & 0 & 1 & 0 & 1 & 1 & 1 & 0.196364 & 0.189405 &
0.437273 & 0.2483090 & 120 & 1229 & 1349 \\
4 & 2011-01-04 & 1 & 0 & 1 & 0 & 2 & 1 & 1 & 0.200000 & 0.212122 &
0.590435 & 0.1602960 & 108 & 1454 & 1562 \\
5 & 2011-01-05 & 1 & 0 & 1 & 0 & 3 & 1 & 1 & 0.226957 & 0.229270 &
0.436957 & 0.1869000 & 82 & 1518 & 1600 \\
6 & 2011-01-06 & 1 & 0 & 1 & 0 & 4 & 1 & 1 & 0.204348 & 0.233209 &
0.518261 & 0.0895652 & 88 & 1518 & 1606 \\
\end{longtable}

\begin{Shaded}
\begin{Highlighting}[]
\FunctionTok{str}\NormalTok{(dataset)}
\end{Highlighting}
\end{Shaded}

\begin{verbatim}
## spc_tbl_ [731 x 16] (S3: spec_tbl_df/tbl_df/tbl/data.frame)
##  $ instant   : num [1:731] 1 2 3 4 5 6 7 8 9 10 ...
##  $ dteday    : Date[1:731], format: "2011-01-01" "2011-01-02" ...
##  $ season    : num [1:731] 1 1 1 1 1 1 1 1 1 1 ...
##  $ yr        : num [1:731] 0 0 0 0 0 0 0 0 0 0 ...
##  $ mnth      : num [1:731] 1 1 1 1 1 1 1 1 1 1 ...
##  $ holiday   : num [1:731] 0 0 0 0 0 0 0 0 0 0 ...
##  $ weekday   : num [1:731] 6 0 1 2 3 4 5 6 0 1 ...
##  $ workingday: num [1:731] 0 0 1 1 1 1 1 0 0 1 ...
##  $ weathersit: num [1:731] 2 2 1 1 1 1 2 2 1 1 ...
##  $ temp      : num [1:731] 0.344 0.363 0.196 0.2 0.227 ...
##  $ atemp     : num [1:731] 0.364 0.354 0.189 0.212 0.229 ...
##  $ hum       : num [1:731] 0.806 0.696 0.437 0.59 0.437 ...
##  $ windspeed : num [1:731] 0.16 0.249 0.248 0.16 0.187 ...
##  $ casual    : num [1:731] 331 131 120 108 82 88 148 68 54 41 ...
##  $ registered: num [1:731] 654 670 1229 1454 1518 ...
##  $ cnt       : num [1:731] 985 801 1349 1562 1600 ...
##  - attr(*, "spec")=
##   .. cols(
##   ..   instant = col_double(),
##   ..   dteday = col_date(format = ""),
##   ..   season = col_double(),
##   ..   yr = col_double(),
##   ..   mnth = col_double(),
##   ..   holiday = col_double(),
##   ..   weekday = col_double(),
##   ..   workingday = col_double(),
##   ..   weathersit = col_double(),
##   ..   temp = col_double(),
##   ..   atemp = col_double(),
##   ..   hum = col_double(),
##   ..   windspeed = col_double(),
##   ..   casual = col_double(),
##   ..   registered = col_double(),
##   ..   cnt = col_double()
##   .. )
##  - attr(*, "problems")=<externalptr>
\end{verbatim}

\begin{Shaded}
\begin{Highlighting}[]
\FunctionTok{dim}\NormalTok{(dataset)}
\end{Highlighting}
\end{Shaded}

\begin{verbatim}
## [1] 731  16
\end{verbatim}

\begin{Shaded}
\begin{Highlighting}[]
\FunctionTok{colnames}\NormalTok{(dataset)}
\end{Highlighting}
\end{Shaded}

\begin{verbatim}
##  [1] "instant"    "dteday"     "season"     "yr"         "mnth"      
##  [6] "holiday"    "weekday"    "workingday" "weathersit" "temp"      
## [11] "atemp"      "hum"        "windspeed"  "casual"     "registered"
## [16] "cnt"
\end{verbatim}

\begin{Shaded}
\begin{Highlighting}[]
\FunctionTok{sum}\NormalTok{(}\FunctionTok{is.na}\NormalTok{(dataset))}
\end{Highlighting}
\end{Shaded}

\begin{verbatim}
## [1] 0
\end{verbatim}

\subsection{Reversing normalization}\label{reversing-normalization}

Normalization is performed using the formula:

\[
\text{Normalized Value} = \frac{\text{Value} - \text{min}}{\text{max} - \text{min}}
\]

Where: - \(\text{Value}\) is the actual value. - \(\text{min}\) is the
minimum value in the range. - \(\text{max}\) is the maximum value in the
range.

To reverse the normalization and retrieve the actual values, we use the
formula:

\[
\text{Value} = (\text{Normalized Value} \times (\text{max} - \text{min})) + \text{min}
\]

Where: - \(\text{Normalized Value}\) is the value after normalization. -
\(\text{min}\) and \(\text{max}\) are as defined above.

The details of the normalization is mentioned in the dataset description
and the abstract

\begin{Shaded}
\begin{Highlighting}[]
\CommentTok{\# Denormalize temperature (temp)}
\CommentTok{\# temp\_min = {-}8, temp\_max = 39}
\NormalTok{dataset}\SpecialCharTok{$}\NormalTok{temp\_actual }\OtherTok{=}\NormalTok{ (dataset}\SpecialCharTok{$}\NormalTok{temp }\SpecialCharTok{*}\NormalTok{ (}\DecValTok{39} \SpecialCharTok{{-}}\NormalTok{ (}\SpecialCharTok{{-}}\DecValTok{8}\NormalTok{))) }\SpecialCharTok{+}\NormalTok{ (}\SpecialCharTok{{-}}\DecValTok{8}\NormalTok{)}

\CommentTok{\# Denormalize feeling temperature (atemp)}
\CommentTok{\# atemp\_min = {-}16, atemp\_max = 50}
\NormalTok{dataset}\SpecialCharTok{$}\NormalTok{atemp\_actual }\OtherTok{=}\NormalTok{ (dataset}\SpecialCharTok{$}\NormalTok{atemp }\SpecialCharTok{*}\NormalTok{ (}\DecValTok{50} \SpecialCharTok{{-}}\NormalTok{ (}\SpecialCharTok{{-}}\DecValTok{16}\NormalTok{))) }\SpecialCharTok{+}\NormalTok{ (}\SpecialCharTok{{-}}\DecValTok{16}\NormalTok{)}

\CommentTok{\# Denormalize humidity (hum)}
\CommentTok{\# Since it\textquotesingle{}s divided by 100, we just multiply by 100}
\NormalTok{dataset}\SpecialCharTok{$}\NormalTok{hum\_actual }\OtherTok{=}\NormalTok{ dataset}\SpecialCharTok{$}\NormalTok{hum }\SpecialCharTok{*} \DecValTok{100}

\CommentTok{\# Denormalize wind speed (windspeed)}
\CommentTok{\# windspeed\_max = 67}
\NormalTok{dataset}\SpecialCharTok{$}\NormalTok{windspeed\_actual }\OtherTok{=}\NormalTok{ dataset}\SpecialCharTok{$}\NormalTok{windspeed }\SpecialCharTok{*} \DecValTok{67}

\CommentTok{\# View the first few rows of the dataset to confirm the changes}
\FunctionTok{kable}\NormalTok{(}\FunctionTok{head}\NormalTok{(dataset)}
\NormalTok{)}
\end{Highlighting}
\end{Shaded}

\begin{longtable}[]{@{}
  >{\raggedleft\arraybackslash}p{(\columnwidth - 38\tabcolsep) * \real{0.0432}}
  >{\raggedright\arraybackslash}p{(\columnwidth - 38\tabcolsep) * \real{0.0595}}
  >{\raggedleft\arraybackslash}p{(\columnwidth - 38\tabcolsep) * \real{0.0378}}
  >{\raggedleft\arraybackslash}p{(\columnwidth - 38\tabcolsep) * \real{0.0162}}
  >{\raggedleft\arraybackslash}p{(\columnwidth - 38\tabcolsep) * \real{0.0270}}
  >{\raggedleft\arraybackslash}p{(\columnwidth - 38\tabcolsep) * \real{0.0432}}
  >{\raggedleft\arraybackslash}p{(\columnwidth - 38\tabcolsep) * \real{0.0432}}
  >{\raggedleft\arraybackslash}p{(\columnwidth - 38\tabcolsep) * \real{0.0595}}
  >{\raggedleft\arraybackslash}p{(\columnwidth - 38\tabcolsep) * \real{0.0595}}
  >{\raggedleft\arraybackslash}p{(\columnwidth - 38\tabcolsep) * \real{0.0486}}
  >{\raggedleft\arraybackslash}p{(\columnwidth - 38\tabcolsep) * \real{0.0486}}
  >{\raggedleft\arraybackslash}p{(\columnwidth - 38\tabcolsep) * \real{0.0486}}
  >{\raggedleft\arraybackslash}p{(\columnwidth - 38\tabcolsep) * \real{0.0541}}
  >{\raggedleft\arraybackslash}p{(\columnwidth - 38\tabcolsep) * \real{0.0378}}
  >{\raggedleft\arraybackslash}p{(\columnwidth - 38\tabcolsep) * \real{0.0595}}
  >{\raggedleft\arraybackslash}p{(\columnwidth - 38\tabcolsep) * \real{0.0270}}
  >{\raggedleft\arraybackslash}p{(\columnwidth - 38\tabcolsep) * \real{0.0649}}
  >{\raggedleft\arraybackslash}p{(\columnwidth - 38\tabcolsep) * \real{0.0703}}
  >{\raggedleft\arraybackslash}p{(\columnwidth - 38\tabcolsep) * \real{0.0595}}
  >{\raggedleft\arraybackslash}p{(\columnwidth - 38\tabcolsep) * \real{0.0919}}@{}}
\toprule\noalign{}
\begin{minipage}[b]{\linewidth}\raggedleft
instant
\end{minipage} & \begin{minipage}[b]{\linewidth}\raggedright
dteday
\end{minipage} & \begin{minipage}[b]{\linewidth}\raggedleft
season
\end{minipage} & \begin{minipage}[b]{\linewidth}\raggedleft
yr
\end{minipage} & \begin{minipage}[b]{\linewidth}\raggedleft
mnth
\end{minipage} & \begin{minipage}[b]{\linewidth}\raggedleft
holiday
\end{minipage} & \begin{minipage}[b]{\linewidth}\raggedleft
weekday
\end{minipage} & \begin{minipage}[b]{\linewidth}\raggedleft
workingday
\end{minipage} & \begin{minipage}[b]{\linewidth}\raggedleft
weathersit
\end{minipage} & \begin{minipage}[b]{\linewidth}\raggedleft
temp
\end{minipage} & \begin{minipage}[b]{\linewidth}\raggedleft
atemp
\end{minipage} & \begin{minipage}[b]{\linewidth}\raggedleft
hum
\end{minipage} & \begin{minipage}[b]{\linewidth}\raggedleft
windspeed
\end{minipage} & \begin{minipage}[b]{\linewidth}\raggedleft
casual
\end{minipage} & \begin{minipage}[b]{\linewidth}\raggedleft
registered
\end{minipage} & \begin{minipage}[b]{\linewidth}\raggedleft
cnt
\end{minipage} & \begin{minipage}[b]{\linewidth}\raggedleft
temp\_actual
\end{minipage} & \begin{minipage}[b]{\linewidth}\raggedleft
atemp\_actual
\end{minipage} & \begin{minipage}[b]{\linewidth}\raggedleft
hum\_actual
\end{minipage} & \begin{minipage}[b]{\linewidth}\raggedleft
windspeed\_actual
\end{minipage} \\
\midrule\noalign{}
\endhead
\bottomrule\noalign{}
\endlastfoot
1 & 2011-01-01 & 1 & 0 & 1 & 0 & 6 & 0 & 2 & 0.344167 & 0.363625 &
0.805833 & 0.1604460 & 331 & 654 & 985 & 8.175849 & 7.999250 & 80.5833 &
10.749882 \\
2 & 2011-01-02 & 1 & 0 & 1 & 0 & 0 & 0 & 2 & 0.363478 & 0.353739 &
0.696087 & 0.2485390 & 131 & 670 & 801 & 9.083466 & 7.346774 & 69.6087 &
16.652113 \\
3 & 2011-01-03 & 1 & 0 & 1 & 0 & 1 & 1 & 1 & 0.196364 & 0.189405 &
0.437273 & 0.2483090 & 120 & 1229 & 1349 & 1.229108 & -3.499270 &
43.7273 & 16.636703 \\
4 & 2011-01-04 & 1 & 0 & 1 & 0 & 2 & 1 & 1 & 0.200000 & 0.212122 &
0.590435 & 0.1602960 & 108 & 1454 & 1562 & 1.400000 & -1.999948 &
59.0435 & 10.739832 \\
5 & 2011-01-05 & 1 & 0 & 1 & 0 & 3 & 1 & 1 & 0.226957 & 0.229270 &
0.436957 & 0.1869000 & 82 & 1518 & 1600 & 2.666979 & -0.868180 & 43.6957
& 12.522300 \\
6 & 2011-01-06 & 1 & 0 & 1 & 0 & 4 & 1 & 1 & 0.204348 & 0.233209 &
0.518261 & 0.0895652 & 88 & 1518 & 1606 & 1.604356 & -0.608206 & 51.8261
& 6.000868 \\
\end{longtable}

\subsection{EDA}\label{eda}

\begin{Shaded}
\begin{Highlighting}[]
\FunctionTok{library}\NormalTok{(ggplot2)}
\FunctionTok{library}\NormalTok{(dplyr)}
\FunctionTok{library}\NormalTok{(gridExtra) }\CommentTok{\# For arranging multiple plots}
\end{Highlighting}
\end{Shaded}

\begin{verbatim}
## 
## Attaching package: 'gridExtra'
\end{verbatim}

\begin{verbatim}
## The following object is masked from 'package:dplyr':
## 
##     combine
\end{verbatim}

\begin{Shaded}
\begin{Highlighting}[]
\NormalTok{key\_variables }\OtherTok{\textless{}{-}} \FunctionTok{c}\NormalTok{(}\StringTok{"temp\_actual"}\NormalTok{, }\StringTok{"atemp\_actual"}\NormalTok{, }\StringTok{"hum\_actual"}\NormalTok{,}
                   \StringTok{"windspeed\_actual"}\NormalTok{, }\StringTok{"cnt"}\NormalTok{)}
\NormalTok{statistics }\OtherTok{\textless{}{-}}\NormalTok{ dataset[, key\_variables] }\SpecialCharTok{\%\textgreater{}\%} \FunctionTok{summary}\NormalTok{()}
\NormalTok{statistics}
\end{Highlighting}
\end{Shaded}

\begin{verbatim}
##   temp_actual      atemp_actual       hum_actual    windspeed_actual
##  Min.   :-5.221   Min.   :-10.781   Min.   : 0.00   Min.   : 1.500  
##  1st Qu.: 7.843   1st Qu.:  6.298   1st Qu.:52.00   1st Qu.: 9.042  
##  Median :15.422   Median : 16.124   Median :62.67   Median :12.125  
##  Mean   :15.283   Mean   : 15.307   Mean   :62.79   Mean   :12.763  
##  3rd Qu.:22.805   3rd Qu.: 24.168   3rd Qu.:73.02   3rd Qu.:15.625  
##  Max.   :32.498   Max.   : 39.499   Max.   :97.25   Max.   :34.000  
##       cnt      
##  Min.   :  22  
##  1st Qu.:3152  
##  Median :4548  
##  Mean   :4504  
##  3rd Qu.:5956  
##  Max.   :8714
\end{verbatim}

\begin{Shaded}
\begin{Highlighting}[]
\CommentTok{\# Load necessary libraries}
\FunctionTok{library}\NormalTok{(ggplot2)}
\FunctionTok{library}\NormalTok{(gridExtra) }\CommentTok{\# For arranging plots}

\CommentTok{\# Set a consistent theme}
\NormalTok{base\_theme }\OtherTok{\textless{}{-}} \FunctionTok{theme\_minimal}\NormalTok{() }\SpecialCharTok{+} 
              \FunctionTok{theme}\NormalTok{(}\AttributeTok{plot.title =} \FunctionTok{element\_text}\NormalTok{(}\AttributeTok{hjust =} \FloatTok{0.5}\NormalTok{),}
                    \AttributeTok{axis.text =} \FunctionTok{element\_text}\NormalTok{(}\AttributeTok{size =} \DecValTok{12}\NormalTok{),}
                    \AttributeTok{axis.title =} \FunctionTok{element\_text}\NormalTok{(}\AttributeTok{size =} \DecValTok{14}\NormalTok{))}

\CommentTok{\# Create individual plots with improved aesthetics}
\NormalTok{p1 }\OtherTok{\textless{}{-}} \FunctionTok{ggplot}\NormalTok{(dataset, }\FunctionTok{aes}\NormalTok{(}\AttributeTok{x =}\NormalTok{ temp\_actual)) }\SpecialCharTok{+}
      \FunctionTok{geom\_histogram}\NormalTok{(}\AttributeTok{bins =} \DecValTok{30}\NormalTok{, }\AttributeTok{fill =} \StringTok{"steelblue"}\NormalTok{, }\AttributeTok{color =} \StringTok{"black"}\NormalTok{) }\SpecialCharTok{+}
      \FunctionTok{ggtitle}\NormalTok{(}\StringTok{"Histogram of temp\_actual"}\NormalTok{) }\SpecialCharTok{+}
\NormalTok{      base\_theme}

\NormalTok{p3 }\OtherTok{\textless{}{-}} \FunctionTok{ggplot}\NormalTok{(dataset, }\FunctionTok{aes}\NormalTok{(}\AttributeTok{x =}\NormalTok{ hum\_actual)) }\SpecialCharTok{+}
      \FunctionTok{geom\_histogram}\NormalTok{(}\AttributeTok{bins =} \DecValTok{30}\NormalTok{, }\AttributeTok{fill =} \StringTok{"tomato"}\NormalTok{, }\AttributeTok{color =} \StringTok{"black"}\NormalTok{) }\SpecialCharTok{+}
      \FunctionTok{ggtitle}\NormalTok{(}\StringTok{"Histogram of hum\_actual"}\NormalTok{) }\SpecialCharTok{+}
\NormalTok{      base\_theme}

\NormalTok{p4 }\OtherTok{\textless{}{-}} \FunctionTok{ggplot}\NormalTok{(dataset, }\FunctionTok{aes}\NormalTok{(}\AttributeTok{x =}\NormalTok{ windspeed\_actual)) }\SpecialCharTok{+}
      \FunctionTok{geom\_histogram}\NormalTok{(}\AttributeTok{bins =} \DecValTok{30}\NormalTok{, }\AttributeTok{fill =} \StringTok{"mediumpurple"}\NormalTok{, }\AttributeTok{color =} \StringTok{"black"}\NormalTok{) }\SpecialCharTok{+}
      \FunctionTok{ggtitle}\NormalTok{(}\StringTok{"Histogram of windspeed\_actual"}\NormalTok{) }\SpecialCharTok{+}
\NormalTok{      base\_theme}

\NormalTok{p5 }\OtherTok{\textless{}{-}} \FunctionTok{ggplot}\NormalTok{(dataset, }\FunctionTok{aes}\NormalTok{(}\AttributeTok{x =}\NormalTok{ cnt)) }\SpecialCharTok{+}
      \FunctionTok{geom\_histogram}\NormalTok{(}\AttributeTok{bins =} \DecValTok{30}\NormalTok{, }\AttributeTok{fill =} \StringTok{"sandybrown"}\NormalTok{, }\AttributeTok{color =} \StringTok{"black"}\NormalTok{) }\SpecialCharTok{+}
      \FunctionTok{ggtitle}\NormalTok{(}\StringTok{"Histogram of cnt"}\NormalTok{) }\SpecialCharTok{+}
      \FunctionTok{xlab}\NormalTok{(}\StringTok{"Total Users/Day"}\NormalTok{) }\SpecialCharTok{+}
\NormalTok{      base\_theme}

\CommentTok{\# Arrange the plots in a grid with 2 columns}
\FunctionTok{grid.arrange}\NormalTok{(p1, p3, p4, p5, }\AttributeTok{ncol =} \DecValTok{2}\NormalTok{)}
\end{Highlighting}
\end{Shaded}

\includegraphics{project_files/figure-latex/unnamed-chunk-10-1.pdf}

\begin{itemize}
\item
  \textbf{Temperature (°C)}: Mean = 15.28, Std Dev = 8.60, Min = -5.22,
  Max = 32.50
\item
  \textbf{Feeling Temperature (°C)}: Mean = 15.31, Std Dev = 10.76, Min
  = -10.78, Max = 39.50
\item
  \textbf{Humidity (\%)}: Mean = 62.79, Std Dev = 14.24, Min = 0.00, Max
  = 97.25
\item
  \textbf{Wind Speed}: Mean = 12.76, Std Dev = 5.19, Min = 1.50, Max =
  34.00
\item
  \textbf{Total Bike Rentals}: Mean = 4504.35, Std Dev = 1937.21, Min =
  22, Max = 8714
\end{itemize}

\begin{Shaded}
\begin{Highlighting}[]
\CommentTok{\# Set a color palette}
\NormalTok{point\_color }\OtherTok{\textless{}{-}} \StringTok{"deepskyblue4"}
\NormalTok{line\_color }\OtherTok{\textless{}{-}} \StringTok{"darkorange"}

\CommentTok{\# Create scatter plots with improved aesthetics}
\NormalTok{p1 }\OtherTok{\textless{}{-}} \FunctionTok{ggplot}\NormalTok{(dataset, }\FunctionTok{aes}\NormalTok{(}\AttributeTok{x =}\NormalTok{ temp\_actual, }\AttributeTok{y =}\NormalTok{ cnt)) }\SpecialCharTok{+} 
      \FunctionTok{geom\_point}\NormalTok{(}\AttributeTok{color =}\NormalTok{ point\_color, }\AttributeTok{alpha =} \FloatTok{0.6}\NormalTok{) }\SpecialCharTok{+} 
      \FunctionTok{geom\_smooth}\NormalTok{(}\AttributeTok{method =} \StringTok{"loess"}\NormalTok{, }\AttributeTok{se =} \ConstantTok{FALSE}\NormalTok{, }\AttributeTok{color =}\NormalTok{ line\_color) }\SpecialCharTok{+} 
      \FunctionTok{ggtitle}\NormalTok{(}\StringTok{\textquotesingle{}Temperature vs Total Bike Rentals\textquotesingle{}}\NormalTok{) }\SpecialCharTok{+}
\NormalTok{      base\_theme}

\NormalTok{p2 }\OtherTok{\textless{}{-}} \FunctionTok{ggplot}\NormalTok{(dataset, }\FunctionTok{aes}\NormalTok{(}\AttributeTok{x =}\NormalTok{ atemp\_actual, }\AttributeTok{y =}\NormalTok{ cnt)) }\SpecialCharTok{+} 
      \FunctionTok{geom\_point}\NormalTok{(}\AttributeTok{color =}\NormalTok{ point\_color, }\AttributeTok{alpha =} \FloatTok{0.6}\NormalTok{) }\SpecialCharTok{+} 
      \FunctionTok{geom\_smooth}\NormalTok{(}\AttributeTok{method =} \StringTok{"loess"}\NormalTok{, }\AttributeTok{se =} \ConstantTok{FALSE}\NormalTok{, }\AttributeTok{color =}\NormalTok{ line\_color) }\SpecialCharTok{+} 
      \FunctionTok{ggtitle}\NormalTok{(}\StringTok{\textquotesingle{}Feeling Temperature vs Total Bike Rentals\textquotesingle{}}\NormalTok{) }\SpecialCharTok{+}
\NormalTok{      base\_theme}

\NormalTok{p3 }\OtherTok{\textless{}{-}} \FunctionTok{ggplot}\NormalTok{(dataset, }\FunctionTok{aes}\NormalTok{(}\AttributeTok{x =}\NormalTok{ hum\_actual, }\AttributeTok{y =}\NormalTok{ cnt)) }\SpecialCharTok{+} 
      \FunctionTok{geom\_point}\NormalTok{(}\AttributeTok{color =}\NormalTok{ point\_color, }\AttributeTok{alpha =} \FloatTok{0.6}\NormalTok{) }\SpecialCharTok{+} 
      \FunctionTok{geom\_smooth}\NormalTok{(}\AttributeTok{method =} \StringTok{"loess"}\NormalTok{, }\AttributeTok{se =} \ConstantTok{FALSE}\NormalTok{, }\AttributeTok{color =}\NormalTok{ line\_color) }\SpecialCharTok{+} 
      \FunctionTok{ggtitle}\NormalTok{(}\StringTok{\textquotesingle{}Humidity vs Total Bike Rentals\textquotesingle{}}\NormalTok{) }\SpecialCharTok{+}
\NormalTok{      base\_theme}

\NormalTok{p4 }\OtherTok{\textless{}{-}} \FunctionTok{ggplot}\NormalTok{(dataset, }\FunctionTok{aes}\NormalTok{(}\AttributeTok{x =}\NormalTok{ windspeed\_actual, }\AttributeTok{y =}\NormalTok{ cnt)) }\SpecialCharTok{+} 
      \FunctionTok{geom\_point}\NormalTok{(}\AttributeTok{color =}\NormalTok{ point\_color, }\AttributeTok{alpha =} \FloatTok{0.6}\NormalTok{) }\SpecialCharTok{+} 
      \FunctionTok{geom\_smooth}\NormalTok{(}\AttributeTok{method =} \StringTok{"loess"}\NormalTok{, }\AttributeTok{se =} \ConstantTok{FALSE}\NormalTok{, }\AttributeTok{color =}\NormalTok{ line\_color) }\SpecialCharTok{+} 
      \FunctionTok{ggtitle}\NormalTok{(}\StringTok{\textquotesingle{}Wind Speed vs Total Bike Rentals\textquotesingle{}}\NormalTok{) }\SpecialCharTok{+}
\NormalTok{      base\_theme}

\CommentTok{\# Arrange the plots in a grid with 2 rows}
\FunctionTok{grid.arrange}\NormalTok{(p1, p2, p3, p4, }\AttributeTok{nrow =} \DecValTok{2}\NormalTok{)}
\end{Highlighting}
\end{Shaded}

\begin{verbatim}
## `geom_smooth()` using formula = 'y ~ x'
## `geom_smooth()` using formula = 'y ~ x'
## `geom_smooth()` using formula = 'y ~ x'
## `geom_smooth()` using formula = 'y ~ x'
\end{verbatim}

\includegraphics{project_files/figure-latex/unnamed-chunk-11-1.pdf}

\begin{Shaded}
\begin{Highlighting}[]
\CommentTok{\# Assuming you have a dataframe called \textquotesingle{}data\textquotesingle{}}
\CommentTok{\# Replace \textquotesingle{}value\_column\textquotesingle{}, \textquotesingle{}facet\_column\textquotesingle{}, and \textquotesingle{}data\textquotesingle{} with your actual column names and dataframe}

\NormalTok{dataset }\OtherTok{\textless{}{-}}\NormalTok{ dataset }\SpecialCharTok{\%\textgreater{}\%}
  \FunctionTok{mutate}\NormalTok{(}\AttributeTok{season\_cat =} \FunctionTok{case\_when}\NormalTok{(}
\NormalTok{    season }\SpecialCharTok{==} \DecValTok{1} \SpecialCharTok{\textasciitilde{}} \StringTok{"Winter"}\NormalTok{,}
\NormalTok{    season }\SpecialCharTok{==} \DecValTok{2} \SpecialCharTok{\textasciitilde{}} \StringTok{"Spring"}\NormalTok{,}
\NormalTok{    season }\SpecialCharTok{==} \DecValTok{3} \SpecialCharTok{\textasciitilde{}} \StringTok{"Summer"}\NormalTok{,}
\NormalTok{    season }\SpecialCharTok{==} \DecValTok{4} \SpecialCharTok{\textasciitilde{}} \StringTok{"Fall"}\NormalTok{,}
    \ConstantTok{TRUE} \SpecialCharTok{\textasciitilde{}} \StringTok{"Unknown"}  \CommentTok{\# Optional: for any value that is not 1, 2, 3, or 4}
\NormalTok{  ))}

\NormalTok{dataset }\OtherTok{\textless{}{-}}\NormalTok{ dataset }\SpecialCharTok{\%\textgreater{}\%}
  \FunctionTok{mutate}\NormalTok{(}\AttributeTok{weathersit\_cat =} \FunctionTok{case\_when}\NormalTok{(}
\NormalTok{    weathersit }\SpecialCharTok{==} \DecValTok{1} \SpecialCharTok{\textasciitilde{}} \StringTok{"Clear/Partly cloudy"}\NormalTok{,}
\NormalTok{    weathersit }\SpecialCharTok{==} \DecValTok{2} \SpecialCharTok{\textasciitilde{}} \StringTok{"Mist/Few clouds"}\NormalTok{,}
\NormalTok{    weathersit }\SpecialCharTok{==} \DecValTok{3} \SpecialCharTok{\textasciitilde{}} \StringTok{"Light Snow/Thunderstorm"}\NormalTok{,}
\NormalTok{    weathersit }\SpecialCharTok{==} \DecValTok{4} \SpecialCharTok{\textasciitilde{}} \StringTok{"Heavy Rain/snow"}\NormalTok{,}
    \ConstantTok{TRUE} \SpecialCharTok{\textasciitilde{}} \StringTok{"Unknown"}  \CommentTok{\# Optional: for any value that is not 1, 2, 3, or 4}
\NormalTok{  ))}


\CommentTok{\# Create the violin plot}
\FunctionTok{ggplot}\NormalTok{(dataset, }\FunctionTok{aes}\NormalTok{(}\AttributeTok{x=}\NormalTok{season\_cat, }\AttributeTok{y=}\NormalTok{cnt, }\AttributeTok{fill=}\NormalTok{season\_cat)) }\SpecialCharTok{+} 
  \FunctionTok{geom\_violin}\NormalTok{(}\AttributeTok{trim=}\ConstantTok{FALSE}\NormalTok{) }\SpecialCharTok{+}
  \FunctionTok{geom\_jitter}\NormalTok{(}\AttributeTok{width=}\FloatTok{0.2}\NormalTok{, }\AttributeTok{alpha=}\FloatTok{0.5}\NormalTok{) }\SpecialCharTok{+}  \CommentTok{\# Add jittered data points}
  \FunctionTok{scale\_fill\_brewer}\NormalTok{(}\AttributeTok{palette=}\StringTok{"Pastel1"}\NormalTok{) }\SpecialCharTok{+}  \CommentTok{\# Color scheme}
  \FunctionTok{theme\_minimal}\NormalTok{() }\SpecialCharTok{+} 
  \FunctionTok{theme}\NormalTok{(}\AttributeTok{axis.text.x =} \FunctionTok{element\_text}\NormalTok{(}\AttributeTok{angle =} \DecValTok{45}\NormalTok{, }\AttributeTok{hjust =} \DecValTok{1}\NormalTok{)) }\SpecialCharTok{+}  \CommentTok{\# Rotate X{-}axis labels}
  \FunctionTok{labs}\NormalTok{(}\AttributeTok{title=}\StringTok{"Total Users vs. Season"}\NormalTok{,}
       \AttributeTok{x=}\StringTok{"Season"}\NormalTok{, }\AttributeTok{y=}\StringTok{"Total Users"}\NormalTok{)}
\end{Highlighting}
\end{Shaded}

\includegraphics{project_files/figure-latex/unnamed-chunk-12-1.pdf}

\begin{Shaded}
\begin{Highlighting}[]
\FunctionTok{ggplot}\NormalTok{(dataset, }\FunctionTok{aes}\NormalTok{(}\AttributeTok{x=}\NormalTok{weathersit\_cat, }\AttributeTok{y=}\NormalTok{cnt, }\AttributeTok{fill=}\NormalTok{weathersit\_cat)) }\SpecialCharTok{+} 
  \FunctionTok{geom\_violin}\NormalTok{(}\AttributeTok{trim=}\ConstantTok{FALSE}\NormalTok{) }\SpecialCharTok{+}
  \FunctionTok{geom\_jitter}\NormalTok{(}\AttributeTok{width=}\FloatTok{0.2}\NormalTok{, }\AttributeTok{alpha=}\FloatTok{0.5}\NormalTok{) }\SpecialCharTok{+}  \CommentTok{\# Add jittered data points}
  \FunctionTok{scale\_fill\_brewer}\NormalTok{(}\AttributeTok{palette=}\StringTok{"Set2"}\NormalTok{) }\SpecialCharTok{+}  \CommentTok{\# Color scheme}
  \FunctionTok{theme\_minimal}\NormalTok{() }\SpecialCharTok{+}
  \FunctionTok{theme}\NormalTok{(}\AttributeTok{axis.text.x =} \FunctionTok{element\_text}\NormalTok{(}\AttributeTok{angle =} \DecValTok{45}\NormalTok{, }\AttributeTok{hjust =} \DecValTok{1}\NormalTok{)) }\SpecialCharTok{+}  \CommentTok{\# Rotate X{-}axis labels}
  \FunctionTok{labs}\NormalTok{(}\AttributeTok{title=}\StringTok{"Total Users vs. Weather Situation"}\NormalTok{,}
       \AttributeTok{x=}\StringTok{"Weather Situation"}\NormalTok{, }\AttributeTok{y=}\StringTok{"Total Users"}\NormalTok{)}
\end{Highlighting}
\end{Shaded}

\includegraphics{project_files/figure-latex/unnamed-chunk-12-2.pdf}

\begin{Shaded}
\begin{Highlighting}[]
\FunctionTok{library}\NormalTok{(reshape2)}
\FunctionTok{library}\NormalTok{(ggplot2)}

\CommentTok{\# Assuming your dataset is named \textquotesingle{}dataset\textquotesingle{}}
\NormalTok{correlation\_matrix }\OtherTok{\textless{}{-}} \FunctionTok{cor}\NormalTok{(dataset[, }\FunctionTok{c}\NormalTok{(}\StringTok{"temp\_actual"}\NormalTok{, }\StringTok{"hum\_actual"}\NormalTok{, }\StringTok{"windspeed\_actual"}\NormalTok{, }\StringTok{"holiday"}\NormalTok{, }\StringTok{"weekday"}\NormalTok{, }\StringTok{"workingday"}\NormalTok{, }\StringTok{"cnt"}\NormalTok{)])}

\CommentTok{\# Melting the correlation matrix}
\NormalTok{melted\_correlation\_matrix }\OtherTok{\textless{}{-}} \FunctionTok{melt}\NormalTok{(correlation\_matrix)}

\CommentTok{\# Plotting the heatmap with enhanced axis text}
\FunctionTok{ggplot}\NormalTok{(}\AttributeTok{data =}\NormalTok{ melted\_correlation\_matrix, }\FunctionTok{aes}\NormalTok{(Var1, Var2, }\AttributeTok{fill =}\NormalTok{ value)) }\SpecialCharTok{+}
    \FunctionTok{geom\_tile}\NormalTok{(}\AttributeTok{color =} \StringTok{"black"}\NormalTok{) }\SpecialCharTok{+}
    \FunctionTok{geom\_text}\NormalTok{(}\FunctionTok{aes}\NormalTok{(}\AttributeTok{label =} \FunctionTok{sprintf}\NormalTok{(}\StringTok{"\%.2f"}\NormalTok{, value)), }\AttributeTok{size =} \DecValTok{3}\NormalTok{, }\AttributeTok{vjust =} \DecValTok{1}\NormalTok{) }\SpecialCharTok{+}
    \FunctionTok{scale\_fill\_gradient2}\NormalTok{(}\AttributeTok{low =} \StringTok{"blue"}\NormalTok{, }\AttributeTok{high =} \StringTok{"red"}\NormalTok{, }\AttributeTok{mid =} \StringTok{"white"}\NormalTok{, }\AttributeTok{midpoint =} \DecValTok{0}\NormalTok{) }\SpecialCharTok{+}
    \FunctionTok{theme\_minimal}\NormalTok{() }\SpecialCharTok{+}
    \FunctionTok{theme}\NormalTok{(}\AttributeTok{axis.text.x =} \FunctionTok{element\_text}\NormalTok{(}\AttributeTok{angle =} \DecValTok{45}\NormalTok{, }\AttributeTok{hjust =} \DecValTok{1}\NormalTok{, }\AttributeTok{size =} \DecValTok{12}\NormalTok{, }\AttributeTok{face =} \StringTok{"bold"}\NormalTok{),  }\CommentTok{\# Increased and bolded x{-}axis text}
          \AttributeTok{axis.text.y =} \FunctionTok{element\_text}\NormalTok{(}\AttributeTok{size =} \DecValTok{12}\NormalTok{, }\AttributeTok{face =} \StringTok{"bold"}\NormalTok{),  }\CommentTok{\# Increased and bolded y{-}axis text}
          \AttributeTok{axis.title =} \FunctionTok{element\_text}\NormalTok{(}\AttributeTok{size =} \DecValTok{12}\NormalTok{),  }\CommentTok{\# Adjusted axis title size}
          \AttributeTok{plot.title =} \FunctionTok{element\_text}\NormalTok{(}\AttributeTok{size =} \DecValTok{14}\NormalTok{)) }\SpecialCharTok{+}  \CommentTok{\# Adjusted plot title size}
    \FunctionTok{ggtitle}\NormalTok{(}\StringTok{\textquotesingle{}Correlation Matrix of All Variables\textquotesingle{}}\NormalTok{)}
\end{Highlighting}
\end{Shaded}

\includegraphics{project_files/figure-latex/correlation-heatmap-1.pdf}

\begin{Shaded}
\begin{Highlighting}[]
\CommentTok{\# Creating a boxplot faceted by season}
\FunctionTok{ggplot}\NormalTok{(dataset, }\FunctionTok{aes}\NormalTok{(}\AttributeTok{x =}\NormalTok{ season\_cat, }\AttributeTok{y =}\NormalTok{ cnt, }\AttributeTok{fill =}\NormalTok{ season\_cat)) }\SpecialCharTok{+} 
    \FunctionTok{geom\_boxplot}\NormalTok{() }\SpecialCharTok{+} 
    \FunctionTok{ggtitle}\NormalTok{(}\StringTok{"Bike Rentals by Season"}\NormalTok{) }\SpecialCharTok{+}
    \FunctionTok{xlab}\NormalTok{(}\StringTok{"Season"}\NormalTok{) }\SpecialCharTok{+}
    \FunctionTok{ylab}\NormalTok{(}\StringTok{"Total Bike Rentals"}\NormalTok{) }\SpecialCharTok{+}
    \FunctionTok{theme\_minimal}\NormalTok{() }\SpecialCharTok{+}  \CommentTok{\# Clean and modern theme}
    \FunctionTok{theme}\NormalTok{(}\AttributeTok{plot.title =} \FunctionTok{element\_text}\NormalTok{(}\AttributeTok{hjust =} \FloatTok{0.5}\NormalTok{, }\AttributeTok{size =} \DecValTok{16}\NormalTok{, }\AttributeTok{face =} \StringTok{"bold"}\NormalTok{),}
          \AttributeTok{axis.title =} \FunctionTok{element\_text}\NormalTok{(}\AttributeTok{size =} \DecValTok{14}\NormalTok{),}
          \AttributeTok{axis.text =} \FunctionTok{element\_text}\NormalTok{(}\AttributeTok{size =} \DecValTok{12}\NormalTok{),}
          \AttributeTok{legend.position =} \StringTok{"none"}\NormalTok{)  }\CommentTok{\# Remove legend as fill color is same as x{-}axis}
\end{Highlighting}
\end{Shaded}

\includegraphics{project_files/figure-latex/unnamed-chunk-13-1.pdf}

\begin{Shaded}
\begin{Highlighting}[]
\FunctionTok{library}\NormalTok{(ggplot2)}

\FunctionTok{ggplot}\NormalTok{(dataset, }\FunctionTok{aes}\NormalTok{(}\AttributeTok{x =}\NormalTok{ weathersit\_cat, }\AttributeTok{y =}\NormalTok{ cnt, }\AttributeTok{fill =}\NormalTok{ weathersit\_cat)) }\SpecialCharTok{+} 
    \FunctionTok{geom\_boxplot}\NormalTok{() }\SpecialCharTok{+} 
    \FunctionTok{ggtitle}\NormalTok{(}\StringTok{"Bike Rentals by Weather Situation"}\NormalTok{) }\SpecialCharTok{+}
    \FunctionTok{xlab}\NormalTok{(}\StringTok{"Weather Situation"}\NormalTok{) }\SpecialCharTok{+}
    \FunctionTok{ylab}\NormalTok{(}\StringTok{"Total Bike Rentals"}\NormalTok{) }\SpecialCharTok{+}
    \FunctionTok{theme\_minimal}\NormalTok{() }\SpecialCharTok{+}  \CommentTok{\# Applying a clean and modern theme}
    \FunctionTok{theme}\NormalTok{(}\AttributeTok{plot.title =} \FunctionTok{element\_text}\NormalTok{(}\AttributeTok{hjust =} \FloatTok{0.5}\NormalTok{, }\AttributeTok{size =} \DecValTok{16}\NormalTok{, }\AttributeTok{face =} \StringTok{"bold"}\NormalTok{),}
          \AttributeTok{axis.title =} \FunctionTok{element\_text}\NormalTok{(}\AttributeTok{size =} \DecValTok{14}\NormalTok{),}
          \AttributeTok{axis.text.x =} \FunctionTok{element\_text}\NormalTok{(}\AttributeTok{angle =} \DecValTok{45}\NormalTok{, }\AttributeTok{hjust =} \DecValTok{1}\NormalTok{, }\AttributeTok{size =} \DecValTok{12}\NormalTok{),  }\CommentTok{\# Slanting x{-}axis text}
          \AttributeTok{axis.text.y =} \FunctionTok{element\_text}\NormalTok{(}\AttributeTok{size =} \DecValTok{12}\NormalTok{),}
          \AttributeTok{strip.text =} \FunctionTok{element\_text}\NormalTok{(}\AttributeTok{size =} \DecValTok{12}\NormalTok{, }\AttributeTok{face =} \StringTok{"bold"}\NormalTok{), }\CommentTok{\# Enhancing facet labels}
          \AttributeTok{legend.position =} \StringTok{"none"}\NormalTok{)  }\CommentTok{\# Removing the legend}
\end{Highlighting}
\end{Shaded}

\includegraphics{project_files/figure-latex/unnamed-chunk-14-1.pdf}

\begin{Shaded}
\begin{Highlighting}[]
\CommentTok{\# Enhanced histogram plot with faceting}
\FunctionTok{ggplot}\NormalTok{(dataset, }\FunctionTok{aes}\NormalTok{(}\AttributeTok{x =}\NormalTok{ cnt)) }\SpecialCharTok{+} 
    \FunctionTok{geom\_histogram}\NormalTok{(}\AttributeTok{bins =} \DecValTok{30}\NormalTok{, }\AttributeTok{fill =} \StringTok{"skyblue"}\NormalTok{, }\AttributeTok{color =} \StringTok{"black"}\NormalTok{) }\SpecialCharTok{+} 
    \FunctionTok{facet\_wrap}\NormalTok{(}\SpecialCharTok{\textasciitilde{}}\NormalTok{workingday, }\AttributeTok{labeller =} \FunctionTok{labeller}\NormalTok{(}\AttributeTok{workingday =} \FunctionTok{c}\NormalTok{(}\StringTok{\textquotesingle{}0\textquotesingle{}} \OtherTok{=} \StringTok{\textquotesingle{}Holiday\textquotesingle{}}\NormalTok{, }\StringTok{\textquotesingle{}1\textquotesingle{}} \OtherTok{=} \StringTok{\textquotesingle{}Working Day\textquotesingle{}}\NormalTok{))) }\SpecialCharTok{+}  \CommentTok{\# Faceting with clear labels}
    \FunctionTok{ggtitle}\NormalTok{(}\StringTok{"Bike Rentals on Working Days vs. Holidays"}\NormalTok{) }\SpecialCharTok{+}
    \FunctionTok{xlab}\NormalTok{(}\StringTok{"Total Bike Rentals"}\NormalTok{) }\SpecialCharTok{+}
    \FunctionTok{ylab}\NormalTok{(}\StringTok{"Frequency"}\NormalTok{) }\SpecialCharTok{+}
    \FunctionTok{theme\_minimal}\NormalTok{() }\SpecialCharTok{+}  \CommentTok{\# Applying a clean and modern theme}
    \FunctionTok{theme}\NormalTok{(}\AttributeTok{plot.title =} \FunctionTok{element\_text}\NormalTok{(}\AttributeTok{hjust =} \FloatTok{0.5}\NormalTok{, }\AttributeTok{size =} \DecValTok{16}\NormalTok{, }\AttributeTok{face =} \StringTok{"bold"}\NormalTok{),}
          \AttributeTok{axis.title =} \FunctionTok{element\_text}\NormalTok{(}\AttributeTok{size =} \DecValTok{14}\NormalTok{),}
          \AttributeTok{axis.text =} \FunctionTok{element\_text}\NormalTok{(}\AttributeTok{size =} \DecValTok{12}\NormalTok{),}
          \AttributeTok{strip.text =} \FunctionTok{element\_text}\NormalTok{(}\AttributeTok{size =} \DecValTok{12}\NormalTok{, }\AttributeTok{face =} \StringTok{"bold"}\NormalTok{))  }\CommentTok{\# Enhancing facet labels}
\end{Highlighting}
\end{Shaded}

\includegraphics{project_files/figure-latex/unnamed-chunk-15-1.pdf}

\subsection{Hypothesis Testing}\label{hypothesis-testing}

\subsubsection{ANOVA test}\label{anova-test}

\begin{Shaded}
\begin{Highlighting}[]
\CommentTok{\# ANOVA test}
\NormalTok{anova\_result }\OtherTok{\textless{}{-}} \FunctionTok{aov}\NormalTok{(cnt }\SpecialCharTok{\textasciitilde{}} \FunctionTok{factor}\NormalTok{(weathersit), }\AttributeTok{data=}\NormalTok{dataset)}
\FunctionTok{summary}\NormalTok{(anova\_result)}
\end{Highlighting}
\end{Shaded}

\begin{verbatim}
##                     Df    Sum Sq   Mean Sq F value Pr(>F)    
## factor(weathersit)   2 2.716e+08 135822286   40.07 <2e-16 ***
## Residuals          728 2.468e+09   3389960                   
## ---
## Signif. codes:  0 '***' 0.001 '**' 0.01 '*' 0.05 '.' 0.1 ' ' 1
\end{verbatim}

\begin{itemize}
\item
  \textbf{Sum of Squares (sum\_sq)}: The between-groups sum of squares
  (for weathersit) is approximately 2.716×10\^{}8, and the within-groups
  sum of squares (Residual) is about 2.468×10\^{}9
\item
  \textbf{Degrees of Freedom (df)}: There are 2 degrees of freedom for
  weathersit (since it has 3 categories: 1, 2, and 3) and 728 degrees of
  freedom for the residuals.
\item
  \textbf{F-statistic (F)}: The calculated F-statistic value is
  approximately 40.07.
\item
  \textbf{P-value (PR(\textgreater F))}: The P-value is extremely low
  (3.106×10\^{}−17).
\end{itemize}

\subsubsection{\texorpdfstring{\textbf{Interpretation:}}{Interpretation:}}\label{interpretation}

\begin{itemize}
\item
  The low P-value (much less than 0.05) suggests that there are
  statistically significant differences in the average number of bike
  rentals (cnt) among different weather situations (clear/few clouds,
  mist/cloudy, light snow/rain).
\item
  This result supports the hypothesis that weather conditions have a
  significant impact on the number of daily active users on the
  bike-sharing platform. Specifically, it indicates that the number of
  users varies significantly with different weather conditions.
\end{itemize}

\begin{Shaded}
\begin{Highlighting}[]
\FunctionTok{library}\NormalTok{(car)}
\end{Highlighting}
\end{Shaded}

\begin{verbatim}
## Loading required package: carData
\end{verbatim}

\begin{verbatim}
## 
## Attaching package: 'car'
\end{verbatim}

\begin{verbatim}
## The following object is masked from 'package:dplyr':
## 
##     recode
\end{verbatim}

\begin{Shaded}
\begin{Highlighting}[]
\FunctionTok{qqPlot}\NormalTok{(anova\_result, }\AttributeTok{id=}\ConstantTok{FALSE}\NormalTok{)}
\end{Highlighting}
\end{Shaded}

\includegraphics{project_files/figure-latex/unnamed-chunk-17-1.pdf}

\begin{Shaded}
\begin{Highlighting}[]
\FunctionTok{leveneTest}\NormalTok{(cnt }\SpecialCharTok{\textasciitilde{}} \FunctionTok{factor}\NormalTok{(weathersit), }\AttributeTok{data=}\NormalTok{dataset)}
\end{Highlighting}
\end{Shaded}

\begin{verbatim}
## Levene's Test for Homogeneity of Variance (center = median)
##        Df F value  Pr(>F)  
## group   2  2.9675 0.05205 .
##       728                  
## ---
## Signif. codes:  0 '***' 0.001 '**' 0.01 '*' 0.05 '.' 0.1 ' ' 1
\end{verbatim}

\subsubsection{\texorpdfstring{\textbf{Analysis and
Interpretation:}}{Analysis and Interpretation:}}\label{analysis-and-interpretation}

\paragraph{1. Boxplot of Bike Rentals for Different Weather
Conditions}\label{boxplot-of-bike-rentals-for-different-weather-conditions}

The boxplot visualizes the distribution of daily bike rentals (cnt)
across different weather conditions. It can be observed that the median
and spread of rentals vary with the weather, suggesting a potential
impact of weather on bike rentals.

\paragraph{2. QQ Plot of Residuals}\label{qq-plot-of-residuals}

The QQ Plot (Quantile-Quantile Plot) of the residuals from the ANOVA
model is used to check the normality of residuals, which is an
assumption of ANOVA. The plot shows how closely the residuals follow a
normal distribution. In this case, the residuals mostly follow the line,
indicating a reasonable approximation to normality, though there are
some deviations at the tails.

\paragraph{3. Levene's Test for Equality of
Variances}\label{levenes-test-for-equality-of-variances}

Levene's Test result:

\begin{itemize}
\item
  Statistic: 2.9675
\item
  P-value: 0.05205
\end{itemize}

Levene's Test is used to assess the equality of variances for a variable
calculated for two or more groups. In this case, it tests whether the
variance in daily bike rentals is the same across different weather
conditions.

\paragraph{Interpretation:}\label{interpretation-1}

\begin{itemize}
\item
  The P-value is marginally above 0.05, suggesting a borderline result
  regarding the equality of variances assumption.
\item
  While it's not a clear violation, this borderline result warrants
  cautious interpretation of the ANOVA results, as ANOVA assumes equal
  variances across groups.
\end{itemize}

\subsubsection{T test between Group 1 (weather situations 1 and 2) and
Group 2 (weather situation 3) clear weather
vs.~rainy/snowy}\label{t-test-between-group-1-weather-situations-1-and-2-and-group-2-weather-situation-3-clear-weather-vs.-rainysnowy}

\begin{Shaded}
\begin{Highlighting}[]
\CommentTok{\# Grouping weather situations}
\NormalTok{group\_1\_2 }\OtherTok{\textless{}{-}} \FunctionTok{subset}\NormalTok{(dataset, weathersit }\SpecialCharTok{\%in\%} \FunctionTok{c}\NormalTok{(}\DecValTok{1}\NormalTok{, }\DecValTok{2}\NormalTok{))}\SpecialCharTok{$}\NormalTok{cnt}
\NormalTok{group\_3 }\OtherTok{\textless{}{-}} \FunctionTok{subset}\NormalTok{(dataset, weathersit }\SpecialCharTok{==} \DecValTok{3}\NormalTok{)}\SpecialCharTok{$}\NormalTok{cnt}

\CommentTok{\# Conducting the t{-}test}
\CommentTok{\# Assuming unequal variances (Welch Two Sample t{-}test)}
\NormalTok{t\_test\_result }\OtherTok{\textless{}{-}} \FunctionTok{t.test}\NormalTok{(group\_1\_2, group\_3, }\AttributeTok{var.equal =} \ConstantTok{FALSE}\NormalTok{)}

\CommentTok{\# Displaying the results}
\FunctionTok{print}\NormalTok{(t\_test\_result)}
\end{Highlighting}
\end{Shaded}

\begin{verbatim}
## 
##  Welch Two Sample t-test
## 
## data:  group_1_2 and group_3
## t = 9.937, df = 22.86, p-value = 9.166e-10
## alternative hypothesis: true difference in means is not equal to 0
## 95 percent confidence interval:
##  2201.828 3360.080
## sample estimates:
## mean of x mean of y 
##  4584.239  1803.286
\end{verbatim}

The t-test comparing the number of bike rentals (cnt) between two
groups---Group 1 (weather situations 1 and 2) and Group 2 (weather
situation 3)---yields the following results:

\begin{itemize}
\item
  \textbf{T-statistic}: 9.937
\item
  \textbf{P-value}: 9.166×10\^{}−10
\end{itemize}

\subsubsection{\texorpdfstring{\textbf{Interpretation:}}{Interpretation:}}\label{interpretation-2}

\begin{itemize}
\item
  The T-statistic is significantly high, and the P-value is extremely
  low (much less than 0.05), indicating a statistically significant
  difference in the number of bike rentals between the two groups.
\item
  This result supports the hypothesis that the number of daily active
  users on the bike-sharing platform decreases significantly with
  worsening weather conditions. Specifically, it shows that the number
  of rentals is significantly lower in weather situation 3 (light snow,
  light rain + thunderstorm + scattered clouds, light rain + scattered
  clouds) compared to weather situations 1 and 2 (clear, few clouds,
  partly cloudy, misty conditions).
\end{itemize}

\subsubsection{}\label{section}

\subsection{Good weather vs Bad
weather}\label{good-weather-vs-bad-weather}

Group 1: Good weather (season 2 and 3 - spring and summer)

Group 2: Bad weather (season 1 - winter

\begin{Shaded}
\begin{Highlighting}[]
\CommentTok{\# Assuming your dataset is loaded into a variable named \textasciigrave{}data\textasciigrave{}}

\CommentTok{\# Subset data for good weather (seasons 2 and 3)}
\NormalTok{good\_weather }\OtherTok{\textless{}{-}} \FunctionTok{subset}\NormalTok{(dataset, season }\SpecialCharTok{\%in\%} \FunctionTok{c}\NormalTok{(}\DecValTok{2}\NormalTok{, }\DecValTok{3}\NormalTok{))}\SpecialCharTok{$}\NormalTok{cnt}

\CommentTok{\# Subset data for bad weather (season 1)}
\NormalTok{bad\_weather }\OtherTok{\textless{}{-}} \FunctionTok{subset}\NormalTok{(dataset, season }\SpecialCharTok{==} \DecValTok{1}\NormalTok{)}\SpecialCharTok{$}\NormalTok{cnt}

\CommentTok{\# Perform the t{-}test}
\NormalTok{t\_test\_result }\OtherTok{\textless{}{-}} \FunctionTok{t.test}\NormalTok{(good\_weather, bad\_weather, }\AttributeTok{var.equal =} \ConstantTok{FALSE}\NormalTok{)}

\CommentTok{\# Print the result}
\FunctionTok{print}\NormalTok{(t\_test\_result)}
\end{Highlighting}
\end{Shaded}

\begin{verbatim}
## 
##  Welch Two Sample t-test
## 
## data:  good_weather and bad_weather
## t = 20.361, df = 405.38, p-value < 2.2e-16
## alternative hypothesis: true difference in means is not equal to 0
## 95 percent confidence interval:
##  2455.30 2980.08
## sample estimates:
## mean of x mean of y 
##  5321.823  2604.133
\end{verbatim}

The t-test comparing the number of bike rentals (cnt) between two
groups---Group 1 (good weather: spring and summer) and Group 2 (bad
weather: winter)---yields the following results:

\begin{itemize}
\item
  \textbf{T-statistic}: 20.361
\item
  \textbf{P-value}: 2.2e-16
\end{itemize}

\subsubsection{\texorpdfstring{\textbf{Interpretation:}}{Interpretation:}}\label{interpretation-3}

\begin{itemize}
\item
  The T-statistic is notably high, and the P-value is extremely low (far
  less than 0.05), indicating a statistically significant difference in
  the number of bike rentals between the two groups.
\item
  This result strongly supports the hypothesis that the number of daily
  active users on the bike-sharing platform is significantly higher in
  good weather conditions (spring and summer) compared to bad weather
  conditions (winter).
\end{itemize}

This analysis reaffirms that seasonal weather variations have a
substantial impact on bike-sharing usage patterns, with more favorable
weather conditions (like in spring and summer) leading to increased bike
rentals.

\subsection{Regression}\label{regression}

\begin{Shaded}
\begin{Highlighting}[]
\FunctionTok{library}\NormalTok{(dplyr)}

\CommentTok{\# Dropping unnecessary columns and normalized columns}
\NormalTok{bike\_data\_for\_regression }\OtherTok{\textless{}{-}}\NormalTok{ dataset }\SpecialCharTok{\%\textgreater{}\%}
  \FunctionTok{select}\NormalTok{(}\SpecialCharTok{{-}}\FunctionTok{c}\NormalTok{(registered, casual, instant, dteday, temp, atemp,yr,mnth, hum, windspeed,weathersit,weathersit\_cat,season,season\_cat,atemp\_actual))}

\CommentTok{\# Splitting the data into X (predictors) and y (response)}
\NormalTok{X }\OtherTok{\textless{}{-}}\NormalTok{ bike\_data\_for\_regression }\SpecialCharTok{\%\textgreater{}\%} \FunctionTok{select}\NormalTok{(}\SpecialCharTok{{-}}\NormalTok{cnt)}
\NormalTok{y }\OtherTok{\textless{}{-}}\NormalTok{ bike\_data\_for\_regression}\SpecialCharTok{$}\NormalTok{cnt}
\end{Highlighting}
\end{Shaded}

\begin{Shaded}
\begin{Highlighting}[]
\FunctionTok{library}\NormalTok{(caret)}
\end{Highlighting}
\end{Shaded}

\begin{verbatim}
## Loading required package: lattice
\end{verbatim}

\begin{Shaded}
\begin{Highlighting}[]
\FunctionTok{set.seed}\NormalTok{(}\DecValTok{42}\NormalTok{)  }\CommentTok{\# For reproducibility}
\NormalTok{trainIndex }\OtherTok{\textless{}{-}} \FunctionTok{createDataPartition}\NormalTok{(y, }\AttributeTok{p =} \FloatTok{0.8}\NormalTok{, }\AttributeTok{list =} \ConstantTok{FALSE}\NormalTok{)}
\NormalTok{X\_train }\OtherTok{\textless{}{-}}\NormalTok{ X[trainIndex, ]}
\NormalTok{X\_test }\OtherTok{\textless{}{-}}\NormalTok{ X[}\SpecialCharTok{{-}}\NormalTok{trainIndex, ]}
\NormalTok{y\_train }\OtherTok{\textless{}{-}}\NormalTok{ y[trainIndex]}
\NormalTok{y\_test }\OtherTok{\textless{}{-}}\NormalTok{ y[}\SpecialCharTok{{-}}\NormalTok{trainIndex]}
\end{Highlighting}
\end{Shaded}

\begin{Shaded}
\begin{Highlighting}[]
\CommentTok{\# Fitting the model}
\NormalTok{regression\_model }\OtherTok{\textless{}{-}} \FunctionTok{lm}\NormalTok{(y\_train }\SpecialCharTok{\textasciitilde{}}\NormalTok{ ., }\AttributeTok{data =}\NormalTok{ X\_train)}

\CommentTok{\# Summary of the model}
\FunctionTok{summary}\NormalTok{(regression\_model)}
\end{Highlighting}
\end{Shaded}

\begin{verbatim}
## 
## Call:
## lm(formula = y_train ~ ., data = X_train)
## 
## Residuals:
##     Min      1Q  Median      3Q     Max 
## -4828.2 -1056.3   -42.4  1036.9  3651.9 
## 
## Coefficients:
##                  Estimate Std. Error t value Pr(>|t|)    
## (Intercept)      4922.888    383.678  12.831  < 2e-16 ***
## holiday          -332.068    345.982  -0.960    0.338    
## weekday            28.161     29.421   0.957    0.339    
## workingday        124.108    130.554   0.951    0.342    
## temp_actual       141.679      6.938  20.421  < 2e-16 ***
## hum_actual        -30.473      4.301  -7.085 4.04e-12 ***
## windspeed_actual  -66.449     11.726  -5.667 2.30e-08 ***
## ---
## Signif. codes:  0 '***' 0.001 '**' 0.01 '*' 0.05 '.' 0.1 ' ' 1
## 
## Residual standard error: 1421 on 580 degrees of freedom
## Multiple R-squared:  0.4639, Adjusted R-squared:  0.4583 
## F-statistic: 83.65 on 6 and 580 DF,  p-value: < 2.2e-16
\end{verbatim}

\begin{Shaded}
\begin{Highlighting}[]
\CommentTok{\# Getting coefficients}
\NormalTok{feature\_weights }\OtherTok{\textless{}{-}} \FunctionTok{coef}\NormalTok{(regression\_model)}

\CommentTok{\# Displaying the coefficients}
\NormalTok{feature\_weights}
\end{Highlighting}
\end{Shaded}

\begin{verbatim}
##      (Intercept)          holiday          weekday       workingday 
##       4922.88795       -332.06812         28.16136        124.10755 
##      temp_actual       hum_actual windspeed_actual 
##        141.67899        -30.47337        -66.44892
\end{verbatim}

\subsubsection{\texorpdfstring{\textbf{Regression Analysis
Summary}}{Regression Analysis Summary}}\label{regression-analysis-summary}

\begin{itemize}
\item
  \textbf{Residuals}: The spread of residuals, ranging from -4828.2 to
  3651.9, indicates the differences between observed and predicted
  values of bike rentals. The median near -42.4 suggests a slight bias
  in underestimation.
\item
  \textbf{Coefficients}:

  \begin{itemize}
  \item
    \textbf{(Intercept) 4922.89}: This is the expected value of cnt when
    all other predictors are 0. A high t-value and a very low p-value
    (\textless{} 2e-16) indicate it's highly significant.
  \item
    \textbf{holiday -332.07}: Suggests bike rentals decrease by 332
    units on holidays, but it's not statistically significant (p-value
    0.338).
  \item
    \textbf{weekday 28.16}: Indicates a slight increase in bike rentals
    depending on the day of the week, but not significant (p-value
    0.339).
  \item
    \textbf{workingday 124.11}: Suggests an increase in bike rentals on
    working days, but again, not statistically significant (p-value
    0.342).
  \item
    \textbf{temp\_actual 141.68}: Shows a significant positive
    relationship between temperature and bike rentals, with an increase
    in temperature leading to more rentals.
  \item
    \textbf{hum\_actual -30.47}: Indicates that higher humidity is
    associated with a decrease in bike rentals, and it's statistically
    significant.
  \item
    \textbf{windspeed\_actual -66.45}: Shows that higher wind speeds are
    associated with fewer bike rentals, and it's also significant.
  \end{itemize}
\end{itemize}

\subsubsection{\texorpdfstring{\textbf{Model
Fit}}{Model Fit}}\label{model-fit}

\begin{itemize}
\item
  \textbf{Residual Standard Error}: 1421 on 580 degrees of freedom. This
  value measures the typical size of the residuals.
\item
  \textbf{Multiple R-squared}: 0.4639. About 46.39\% of the variance in
  bike rental counts is explained by the model, which is a moderate fit.
\item
  \textbf{Adjusted R-squared}: 0.4583. This is a slight adjustment to
  the R-squared value, accounting for the number of predictors.
\item
  \textbf{F-statistic}: 83.65 on 6 and 580 DF, with a p-value
  \textless{} 2.2e-16. This suggests the model as a whole is
  statistically significant.
\end{itemize}

\subsubsection{\texorpdfstring{\textbf{Interpretation}}{Interpretation}}\label{interpretation-4}

\begin{itemize}
\item
  The model suggests that temperature has the most significant positive
  impact on bike rentals, followed by negative impacts from humidity and
  wind speed.
\item
  Variables like holidays, weekdays, and working days are not
  statistically significant in predicting bike rentals in this model.
\item
  The model's moderate R-squared value implies there is room for
  improvement, possibly by including other relevant variables or
  interactions not considered in this model.
\item
  The significance of temperature, humidity, and wind speed aligns with
  intuitive expectations about outdoor activities like bike-sharing.
\end{itemize}

\end{document}
